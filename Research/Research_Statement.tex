%-------------------------
% Resume in Latex
% Author : Jake Gutierrez
% Based off of: https://github.com/sb2nov/resume
% License : MIT
%------------------------

\documentclass[
%letterpaper
article,12pt]{article}

\usepackage{lipsum}
\usepackage{latexsym}
\usepackage[empty]{fullpage}
\usepackage{titlesec}
\usepackage{marvosym}
\usepackage[usenames,dvipsnames]{color}
\usepackage{verbatim}
\usepackage{enumitem}
%\usepackage[hidelinks]{hyperref}
\usepackage{fancyhdr}
\usepackage[english]{babel}
\usepackage{tabularx}
\usepackage{fontawesome5}
\usepackage{simpleicons}
%\usepackage[T1]{fontenc}
\usepackage{multicol}
\usepackage[colorlinks=true, urlcolor=blue, linkcolor=red]{hyperref}
\setlength{\multicolsep}{-3.0pt}
\setlength{\columnsep}{-1pt}
\input{glyphtounicode}

\usepackage{amsmath}
\usepackage{amssymb}
\usepackage{amsthm}

% Define theorem environment
\newtheorem{theorem}{Theorem}

%----------FONT OPTIONS----------
% sans-serif
% \usepackage[sfdefault]{FiraSans}
% \usepackage[sfdefault]{roboto}
% \usepackage[sfdefault]{noto-sans}
% \usepackage[default]{sourcesanspro}

% serif
% \usepackage{CormorantGaramond}
% \usepackage{charter}

\pagestyle{fancy}
\fancyhf{} % clear all header and footer fields
\fancyfoot{}
\renewcommand{\headrulewidth}{0pt}
\renewcommand{\footrulewidth}{0pt}

% Adjust margins
\addtolength{\oddsidemargin}{-0.6in}
\addtolength{\evensidemargin}{-0.5in}
\addtolength{\textwidth}{1.19in}
\addtolength{\topmargin}{-.7in}
\addtolength{\textheight}{1.4in}

\urlstyle{same}

\raggedbottom
\raggedright
%\setlength{\tabcolsep}{0in}

\setlength\parindent{24pt}


% Sections formatting
\titleformat{\section}{
  \vspace{-4pt}\scshape\raggedright\large\bfseries
}{}{0em}{}[\color{black}\titlerule \vspace{-5pt}]

% Ensure that generate pdf is machine readable/ATS parsable
\pdfgentounicode=1

%-------------------------
% Custom commands
\newcommand{\resumeItem}[1]{
  \item\small{
    {#1 \vspace{-2pt}}
  }
}

\newcommand{\classesList}[4]{
    \item\small{
        {#1 #2 #3 #4 \vspace{-2pt}}
  }
}

\newcommand{\resumeSubheading}[4]{
  \vspace{-2pt}\item
    \begin{tabular*}{1.0\textwidth}[t]{l@{\extracolsep{\fill}}r}
      \textbf{#1} & \textbf{\small #2} \\
      \textit{\small#3} & \textit{\small #4} \\
    \end{tabular*}\vspace{-7pt}
}

\newcommand{\resumeSubSubheading}[2]{
    \item
    \begin{tabular*}{0.97\textwidth}{l@{\extracolsep{\fill}}r}
      \textit{\small#1} & \textit{\small #2} \\
    \end{tabular*}\vspace{-7pt}
}

\newcommand{\resumeProjectHeading}[2]{
    \item
    \begin{tabular*}{1.001\textwidth}{l@{\extracolsep{\fill}}r}
      \small#1 & \textbf{\small #2}\\
    \end{tabular*}\vspace{-7pt}
}

\newcommand{\resumeSubItem}[1]{\resumeItem{#1}\vspace{-4pt}}

\renewcommand\labelitemi{$\vcenter{\hbox{\tiny$\bullet$}}$}
\renewcommand\labelitemii{$\vcenter{\hbox{\tiny$\bullet$}}$}

\newcommand{\resumeSubHeadingListStart}{\begin{itemize}[leftmargin=0.0in, label={}]}
\newcommand{\resumeSubHeadingListEnd}{\end{itemize}}
\newcommand{\resumeItemListStart}{\begin{itemize}}
\newcommand{\resumeItemListEnd}{\end{itemize}\vspace{-5pt}}

\newcommand{\sectiono}[1]{\section{#1}\setcounter{equation}{0}}
\newcommand{\subsectiono}[1]{\subsection{#1}}
%\setcounter{equation}{0}
\renewcommand{\theequation}{\thesection.\arabic{equation}}

\pagenumbering{Roman}
%%%%%%%%%%%%%%%%%%%%%%%%%%%%%%%%%%%%

\begin{document}

%----------HEADING----------
% \begin{tabular*}{\textwidth}{l@{\extracolsep{\fill}}r}
%   \textbf{\href{http://sourabhbajaj.com/}{\Large Sourabh Bajaj}} & Email : \href{mailto:sourabh@sourabhbajaj.com}{sourabh@sourabhbajaj.com}\\
%   \href{http://sourabhbajaj.com/}{http://www.sourabhbajaj.com} & Mobile : +1-123-456-7890 \\
% \end{tabular*}

\begin{center}
{\Huge \scshape Research Statement} \\ 
\vspace{0.12in}
{\Large \scshape Partha Mukhopadhyay} \\ 
\vspace{0.07in}
\small{ 
\href{https://www.imsc.res.in/partha_mukhopadhyay}{{\faHome} \hspace{0.3em}} 
\href{mailto:mukhopadhyay.res@gmail.com}{{\faEnvelope} \hspace{0.3em}} 
\href{https://t.me/ParthoM7}{{\faTelegram} \hspace{0.3em}}
\href{https://www.linkedin.com/in/parthom7}{{\faLinkedin} \hspace{0.3em}}
\href{https://www.youtube.com/@ParthoM7}{{\faYoutube} \hspace{0.3em}}
\href{https://inspirehep.net/authors/996534}{{\simpleicon{inspire}} \hspace{0.13em}}
\href{https://github.com/ParthoM7}{{\faGithub}}} \\
\vspace{0.07in}
\today \\
\vspace{0.07in}


\end{center}

\medskip 

Here I briefly describe research work done since around 2018.  Mainly two directions were pursued: (1) Gravity and (2) String/Field/Lattice. Both were originally based on a common theme of understanding relativistic dynamics of compact objects or bound states both at classical and quantum levels. However, as a by-product a new way of constructing lattice theories emerged where space-time symmetries are manifest. This is where the current focus lies\footnote{Most of the work done is still unpublished. Going forward, emphasis will be given on publishing the existing results.}. 

\section{Gravity}
Motion of a compact object in general relativity in probe approximation was described elegantly in terms of gravitational maultipole moments by W. G. Dixon (see review \cite{dixon-rev})\footnote{See \cite{harte} for a review of subsequent works done on going beyond the probe approximation by incorporating self-field effects for a source in a force field. In the context of gravity, this understanding is still incomplete. }. My studies in this context are described below.  

\vspace{.1in}
\noindent
\textbf{A configuration space approach to few-body problem:} The aforementioned works \cite{dixon-rev, harte} considered motion of only one compact object. In order to incorporate multiple compact objects, I attempted to develop a configuration space approach to the problem. In this case the orbit system (i.e. a classical bound state) itself behaves as a compact object to which Dixon's construction must be applicable. It is to be noted here that Dixon's construction has an implicit fiber bundle description. 

In earlier work \cite{semi-classical}, I considered a configuration space approach to study motion of a string. A specific mathematical feature that emerges is the tubular geometry around the submanifold of the motion of the center of mass (CM) embedded in the full configuration space. The details of tubular geometric techniques were developed in \cite{tubular, tubular-gen}. Tubular geometry also has a fiber bundle structure. 

So a natural question to ask is if there is any unified fiber bundle description of both Dixon's system and the aforementioned tubular system. Indeed in a prolonged work, partially in collaboration with Aditya Vaswani (a visiting project student) a unified mathematical tool was developed.  The latter can describe both Dixon's system and the few-body system in different limits. The work required finding a more general description of tubular geometry in the following sense. In the usual description of tubular geometry around a submanifold \cite{tubular}, the fibers are parametrized by the lengths along geodesics that are orthogonal to the submanifold. In the aforementioned generalization, the geodesics are not orthogonal, rather their angles with the submanifold vary smoothly. 

The results of the above analysis are very formal and have not been considered for publication so far. However, a concrete result came out as a by-product. A specific definition of the CM of string was considered in \cite{semi-classical}. This was obtained by geometrically generalizing the usual definition of CM in non-relativistic classical mechanics. Dixon's work also proposes a definition of the CM of a compact object which is well-accepted in the literature\footnote{Dixon showed \cite{dixon-rev} that conservation of stress-tensor of the compact object is equivalent to a generalized version of Newton's law for linear and angular motions along the timelike trajectory of this CM.}. This definition, however, is too restrictive to be usable in a multi-body problem. It turns out that a relaxed version of Dixon's definition does indeed exist which is suitable for the multi-body problem and that it matches exactly with the definition considered in \cite{semi-classical} when applied appropriately. This result may be expected, given that both the approaches are geometrically motivated and it is nice to establish that.

\noindent
{\bf Construction of lightcone multipole moments:} Dixon's multipole moments are defined by taking a spatial slicing of the spacetime and performing integration over the intersection of the spatial slice and the world-tube of the object. When radiation is received from such an object at the observer's location, information from different spatial slices reach at the same time  due to the finite size of the object. Because of this effect the relation between the radiation received and multipole moments of the source involves infinite number of time derivatives. 

In an effort to simplify this relation, I considered a specific lightcone foliation (instead of spatial foliation) of the spacetime so that the received radiation contains instantaneous information about the source multipole moments. Because of the null nature of the lightcone, it can be argued that this is a consistent procedure. Therefore the job is to generalize Dixon's construction in \cite{dixon-rev} to the present situation. This work is half-way through. 

\vspace{.1in}
\noindent
{\bf Future directions:} I would like to complete the lightcone construction first and then look into the relevance of the previous analysis in this context. I am interested in exploring how the analysis of the few body system and its observation can be analyzed using this lightcone framework. 

\section{String/Field/Lattice} 

String quantization in an arbitrary background $ \mathcal{M} $ was developed in the early $80$'s. In this framework, the
classical limit is given by a classical worldsheet (i.e. a 2D extremal surface in $ \mathcal{M} $) and the semi-classical
expansion is obtained from a Riemann normal coordinate expansion in $\mathcal{M}$. However, a different semi-classical
quantization must exist where the classical limit is given by a worldline (i.e. the string becomes point-like). It was
argued in \cite{semi-classical} that the corresponding semi-classical expansion should be obtained from the tubular
expansion around $\mathcal{M}$ sitting as a submanifold within $\mathcal{L}\mathcal{M}$ (loop space of $\mathcal{M}$) which is the full configuration
space of the string. This expansion was found in \cite{cut-off} by first solving the problem in a cut-off loop space
given by,
\begin{align}
  \mathcal{L} \mathcal{M}^{(N)} & \equiv (\mathcal{M}^N)_C ~,
\end{align}
and then taking $N\to \infty$ limit in a suitable manner. Here $\mathcal{M}^N$ is the Cartesian product of $N$ copies of $\mathcal{M}$ and $(\cdots)_C$ indicates a cyclic ordering.

UV divergences appear in the above description in the form of the dimension ($\infty$) of $\mathcal{L}\mathcal{M}$. Therefore, a natural regulator in this case will be $N$ in the above equation. Given that $\mathcal{L}\mathcal{M}^{(N)}$ is the configuration space of an $N$-string-bit system, it is basically a lattice regularization. Over years, lattice regularization has also been considered for worldsheet non-perturbative computations in the context of AdS/CFT (see references in \cite{covlatt1}). A natural question to ask is whether a BRST formulation can be retained at a finite $N$. This would require one to protect the worldsheet symmetries under discretization. 

Loss of space-time symmetries is a standard problem in the usual lattice constructions of QFT's, the technical reason being the use of difference approximation of the derivative operator \cite{wilson}. The same is responsible for the problems like fermion doubling and loss of chirality whose existing solutions are quite complicated. Such a lattice derivative is usually referred to as {\it local}, as it connects a few nearest neighbor sites. General arguments exist \cite{NNtheorem} claiming that the aforementioned problems are unavoidable for local theories. 

A non-local derivative, known as SLAC derivative, was proposed in \cite{slac} which solved the above problems. However,
the authors proved the following theorem,
\begin{theorem}[SLAC Theorem]
	No definition of a gradient operator on the lattice satisfies the Leibniz product rule.
	\label{slac-theorem}
\end{theorem}
As a result Poincaré invarinace or any other space-time symmetries could not be protected on the lattice. While this by
itself is usually not considered a problem, certain investigations \cite{karsten-smit} showed that gauge theories based
on SLAC derivative are unsatisfactory as they are non-local and non-covariant in the continuum limit and the same
conclusion persisted in the literature till date. Therefore, one concludes the following regarding the current state of
affair: \textit{Both local and non-local lattice derivatives/theories fail to solve the problems of fermion doubling, chirality, space-time symmetries (hence supersymmetries) all together at the same time in presence of gauge fields and in arbitrary dimensions with the right continuum limit!}

In \cite{covlatt1} we argued that the spacetime symmetries can be preserved on the lattice if the associated discrete derivative satisfies the same continuum calculus. Over the years, we’ve demonstrated \cite{covlatt1, covlatt2, covlatt3} (see \cite{YT}) the existence of two such derivatives - commonly referred to as an {\it exact derivatives} (ED) - and developed a consistent mathematical framework around them. This framework reproduces the full structure of continuum calculus on the lattice, including a lattice version of tempered distribution theory in the spirit of Laurent Schwartz. Using this approach, it was possible to construct the full Poincaré algebra for an arbitrary scalar field theory on the lattice in any dimension. A brief discussion is given below. 

An ED is given by a sum of infinite number of terms involving arbitrary higher powers of forward and backward difference operators. It is also 
SLAC-type in the sense that its action on the lattice Fourier modes produces the same results, modulo certain additional subtleties. However, it 
is not non-local; rather it produces ultra-local theories. The lattice tempered distribution theory is given in terms of ED and its operational 
inverse - {\it a formal lattice integral} (LI). The construction establishes that the system (ED, LI) satisfies the same continuum calculus. A lattice 
theory constructed using (ED, LI) satisfies the following properties.
\begin{itemize}
\item
Although the operations involved are intrinsically different from those in the continuum, the variational calculus of the lattice field theory remains, algorithmically, the same as that of continuum theory.

\item
It follows that the dynamical or constraint equations of the continuum theory remain true for the lattice, site-wise. This is referred to as {\it ultra-locality}.

\item
All infinitesimal symmetries (space-time, gauge, chiral etc) can be formulated. Noether's theorems hold. 

\item
The momentum space theory corresponding to this covariant lattice field theory is given by the same as that of the continuum theory with sharp 
momentum cutoff at the first Brillouin zone, but with certain restrictions on the field configurations. This restriction solves the earlier problem of 
interference between gauge invariance and sharp momentum cutoff. 

\item
Because of the use of LI, the above construction may look formal in position space. However, it's shown in \cite{covlatt2} that LI can be replaced by the standard summation if a suitable finer-lattice-technique is adopted. The latter is similar to the one used by G. Bergner in \cite{no-go} while studying SLAC-type non-local derivatives in the context of low-dimensional Wess-Zumino models. However, Bergner's action still suffers from non-locality for usual gauge theories due to the same problems demonstrated in \cite{karsten-smit}. 

\end{itemize}

A successful construction of fully symmetric lattice QFTs would imply that the UV cutoff is not just a mathematical artifact, but a physical reality. 
This opens the door to a formulation of QFTs that are manifestly UV finite - in essence, a rejection of the continuum hypothesis (CH) in a 
physically meaningful way. This perspective aligns with the observation that CH can never be verified empirically. In fact, the unphysical UV 
divergences in standard QFTs are tied to the implicit assumption of CH in all existing formulations. The presence of CH is rooted in the absence 
of a physical UV cutoff—something a symmetry-preserving lattice can naturally provide.

Our methods can also be extended to a large class of curved backgrounds. A key example is the worldsheet theory of a string in an arbitrary 
curved target space, which possesses local 2D $ (\rm{Diff} \times \rm{Weyl}) $ symmetry. The corresponding lattice formulation \cite{covlatt1} 
provides, for the first time, a fully covariant description of string bits in such backgrounds. In particular in \cite{covlatt1} we demonstrate - at the classical level - both the Virasoro algebra (local) and the target space isometry algebra (global) for the string bits. 

By using the lattice tempered distribution theory of \cite{covlatt2} one is now able to quantize the covariant string bits \cite{q-string-bits}. In flat 
space, one gets the following results: (1) At a finite $N$, there exists a symmetric ordering such that the quantum Virasoro anomaly vanishes in 
arbitrary dimensions. This of course does not have the right field theory limit in the continuum ($N \to \infty$). Nevertheless, it does describe 
relativistic bound state of string bits that freely propagates in a quantum consistent manner. Preliminary analysis suggests that in a curved 
space the quantum consistency requires the background to be Ricci flat. (2) The standard normal ordering prescription, on the other hand, has 
the right field theory limit implying that the Virasoro anomaly vanishes only in $26$ dimensions.

Before ending I would like to mention some aspects of the mathematical construction of lattice tempered distributions. Unlike in the continuum, 
the lattice tempered distributions, whose derivatives satisfy the same distributional identities of the continuum, are also classical (lattice) 
functions. In fact, this offers one way of understanding how the lattice theories can avoid UV divergences and remain covariant at the same 
time. This however, at the same time, implies that all the distributional identities can be directly verified using the explicit expressions of the 
EDs, without taking any help of the abstract approach of using the inner product between a distribution and a test function. Such direct calculations lead to non-trivial identities involving Stirling numbers, Bernoulli numbers and so on. 
%Some of these issues along with various other aspects of the covariant lattice QFTs (see, for instance \href{https://youtube.com/playlist?
%list=PLLgNOPmTPi1_riYtTFPtui_pHjY7hT8Rt&si=Ubg2pZX7ba1ZDJBX}{the YouTube playlist}) are being studied in \cite{covlatt3}. 

Some pre-published discussion on the above can be found in the
\href{https://youtube.com/playlist?list=PLLgNOPmTPi1_riYtTFPtui_pHjY7hT8Rt&si=Ubg2pZX7ba1ZDJBX}{YouTube playlist}. 


\begin{thebibliography}{99}

%\cite{Dixon:2015vxa}
\bibitem{dixon-rev}
W.~G.~Dixon,
``The New Mechanics of Myron Mathisson and Its Subsequent Development,''
Fund. Theor. Phys. \textbf{179} (2015), 1-66
doi:10.1007/978-3-319-18335-0\_1
%17 citations counted in INSPIRE as of 28 Jul 2023

%\cite{Harte:2014wya}
\bibitem{harte}
A.~I.~Harte,
``Motion in classical field theories and the foundations of the self-force problem,''
Fund. Theor. Phys. \textbf{179} (2015), 327-398
doi:10.1007/978-3-319-18335-0\_12
[arXiv:1405.5077 [gr-qc]].
%28 citations counted in INSPIRE as of 28 Jul 2023

    
%\cite{Mukhopadhyay:2012qy}
\bibitem{semi-classical} 
P.~Mukhopadhyay,
``On a semi-classical limit of loop space quantum mechanics,''
ISRN High Energy Phys.\  {\bf 2013}, 398030 (2013)
[arXiv:1202.2735 [hep-th]];
%%CITATION = ARXIV:1202.2735;%%
%1 citations counted in INSPIRE as of 01 Mar 2014

%\cite{Mukhopadhyay:2012tp}
\bibitem{tubular}
P.~Mukhopadhyay,
``All order covariant tubular expansion,''
Rev. Math. Phys. \textbf{26} (2014) no.1, 1350019
doi:10.1142/S0129055X13500190
[arXiv:1203.1151 [gr-qc]];
%3 citations counted in INSPIRE as of 28 Jul 2023

\bibitem{cut-off}
%\cite{Mukhopadhyay:2014dta}
P.~Mukhopadhyay,
``A cut-off tubular geometry of loop space,''
Rev.\ Math.\ Phys.\  {\bf 29}, 1750006 (2017)
  doi:10.1142/S0129055X17500064
arXiv:1407.7355 [hep-th].
%%CITATION = ARXIV:1407.7355;%%

%\cite{Mukhopadhyay:2016qfs}
\bibitem{tubular-gen}
P.~Mukhopadhyay,
``General Construction of Tubular Geometry,''
[arXiv:1608.07506 [hep-th]] \\
%0 citations counted in INSPIRE as of 28 Jul 2023 
%(Unpublished in a journal due to lack of a reviewer)

%\cite{Wilson:1974sk}
\bibitem{wilson}
K.~G.~Wilson,
``Confinement of Quarks,''
Phys. Rev. D \textbf{10} (1974), 2445-2459
doi:10.1103/PhysRevD.10.2445
%6061 citations counted in INSPIRE as of 12 Jul 2023

%\cite{Nielsen:1980rz}
\bibitem{NNtheorem}
H.~B.~Nielsen and M.~Ninomiya,
%``Absence of Neutrinos on a Lattice. 1. Proof by Homotopy Theory,''
Nucl. Phys. B \textbf{185} (1981), 20
[erratum: Nucl. Phys. B \textbf{195} (1982), 541]
doi:10.1016/0550-3213(82)90011-6
%1399 citations counted in INSPIRE as of 13 Jul 2023

%\cite{Nielsen:1981xu}
%\bibitem{Nielsen:1981xu}
H.~B.~Nielsen and M.~Ninomiya,
%``Absence of Neutrinos on a Lattice. 2. Intuitive Topological Proof,''
Nucl. Phys. B \textbf{193} (1981), 173-194
doi:10.1016/0550-3213(81)90524-1
%919 citations counted in INSPIRE as of 13 Jul 2023

%\cite{Drell:1976bq}
\bibitem{slac}
S.~D.~Drell, M.~Weinstein and S.~Yankielowicz,
``Variational Approach to Strong Coupling Field Theory. 1. Phi**4 Theory,''
Phys. Rev. D \textbf{14} (1976), 487
doi:10.1103/PhysRevD.14.487;
%256 citations counted in INSPIRE as of 14 Jul 2023
%\cite{Drell:1976mj}
%\bibitem{Drell:1976mj}
S.~D.~Drell, M.~Weinstein and S.~Yankielowicz,
``Strong Coupling Field Theories. 2. Fermions and Gauge Fields on a Lattice,''
Phys. Rev. D \textbf{14} (1976), 1627
doi:10.1103/PhysRevD.14.1627
%253 citations counted in INSPIRE as of 14 Jul 2023

%\cite{Karsten:1978nb}
\bibitem{karsten-smit}
L.~H.~Karsten and J.~Smit,
``Axial Symmetry in Lattice Theories,''
Nucl. Phys. B \textbf{144} (1978), 536-546
doi:10.1016/0550-3213(78)90385-1;
%88 citations counted in INSPIRE as of 26 Jul 2023
%\cite{Karsten:1979wh}
%\bibitem{karsten-smit}
L.~H.~Karsten and J.~Smit,
``The Vacuum Polarization With {SLAC} Lattice Fermions,''
Phys. Lett. B \textbf{85} (1979), 100-102
doi:10.1016/0370-2693(79)90786-X
%86 citations counted in INSPIRE as of 15 Jul 2023

%\cite{Dondi:1976tx}
\bibitem{dondi}
P.~H.~Dondi and H.~Nicolai,
``Lattice Supersymmetry,''
Nuovo Cim. A \textbf{41}, 1 (1977)
doi:10.1007/BF02730448
%113 citations counted in INSPIRE as of 13 Jul 2023

%\cite{Bergner:2009vg}
\bibitem{no-go}
G.~Bergner,
``Complete supersymmetry on the lattice and a No-Go theorem,''
JHEP \textbf{01} (2010), 024
doi:10.1007/JHEP01(2010)024
[arXiv:0909.4791 [hep-lat]].
%55 citations counted in INSPIRE as of 15 Jul 2023

%\cite{Reisz:1987da}
\bibitem{reisz}
T.~Reisz,
``A Power Counting Theorem for Feynman Integrals on the Lattice,''
Commun. Math. Phys. \textbf{116}, 81 (1988)
doi:10.1007/BF01239027
%94 citations counted in INSPIRE as of 26 Sep 2023
;
%\cite{Reisz:1987pw}
%\bibitem{Reisz:1987pw}
T.~Reisz,
``A Convergence Theorem for Lattice Feynman Integrals With Massless Propagators,''
Commun. Math. Phys. \textbf{116}, 573 (1988)
doi:10.1007/BF01224902
%52 citations counted in INSPIRE as of 26 Sep 2023
;
%\cite{Reisz:1987px}
%\bibitem{Reisz:1987px}
T.~Reisz,
``Renormalization of Feynman Integrals on the Lattice,''
Commun. Math. Phys. \textbf{117}, 79 (1988)
doi:10.1007/BF01228412
%59 citations counted in INSPIRE as of 26 Sep 2023
;
%\cite{Reisz:1987hx}
%\bibitem{Reisz:1987hx}
T.~Reisz,
``Renormalization of Lattice Feynman Integrals With Massless Propagators,''
Commun. Math. Phys. \textbf{117}, 639 (1988)
doi:10.1007/BF01218390
%41 citations counted in INSPIRE as of 26 Sep 2023
;
%\cite{Reisz:1988kk}
%\bibitem{Reisz:1988kk}
T.~Reisz,
``Lattice Gauge Theory: Renormalization to All Orders in the Loop Expansion,''
Nucl. Phys. B \textbf{318}, 417-463 (1989)
doi:10.1016/0550-3213(89)90613-5
%71 citations counted in INSPIRE as of 26 Sep 2023


%\cite{Mukhopadhyay:2023yfi}
\bibitem{covlatt1}
P.~Mukhopadhyay,
``Covariant Closed String Bits - Classical Theory,''
[arXiv:2307.16520 [hep-th]].
%0 citations counted in INSPIRE as of 01 Aug 2023

\bibitem{covlatt2}
P.~Mukhopadhyay,
``A Proposal for a Symmetry Preserving UV cutoff,'' {\it in preparation}. 

\bibitem{covlatt3}
P.~Mukhopadhyay -  ongoing work on further development of {\it covariant-lattice-cutoff proposal and related mathematics}.

\bibitem{YT}
P.~Mukhopadhyay - some of the yet-to-be published results are discussed in \href{https://youtube.com/playlist?list=PLLgNOPmTPi1_riYtTFPtui_pHjY7hT8Rt&si=gl_7YEJjyM7MilmS}{this YouTube playlist}.

\bibitem{q-string-bits}
P.~Mukhopadhyay,
``Quantum Covariant Closed String Bits,'' {\it in progress}.


\end{thebibliography}





\end{document}




















